\documentclass{llncs}
\usepackage[latin1]{inputenc}
\usepackage{graphicx}        % standard LaTeX graphics tool
\usepackage{url}
\usepackage{listings}

\begin{document}

\title{Taming noisy fitness using Wilcoxon statistical test}
\subtitle{}

\author{J.J. Merelo\inst{1}  \and Antonio
  Mora\inst{1}  \and Anna I. Esparcia-Alc�zar\inst{2} \and V�ctor Rivas\inst{3}}

\institute{University of Granada\\
       Department of Computer Architecture and Technology, ETSIIT\\
       18071 - Granada\\
       \email{\{jmerelo,pedro,amorag\}@geneura.ugr.es}
\and
S2Grupo\\
\email{aiesparcia@s2grupo.com}
\and
Universidad de M�laga\\
Departamento de Lenguajes y Sistemas Inform�ticos\\
\email{ccottap@lcc.uma.es}
}

\maketitle
\begin{abstract}
\end{abstract}

\section{Introduction}


The rest of the paper is organized as follows: next we describe the
state of the art in new and uncanny implementations of evolutionary algorithms and
past implementations using different JS interpreters. The
algorithm that has been adapted to JS is described next along with the
experimental setup in Section \ref{sec:node}; finally results are
presented in Section \ref{sec:dist} and its implications
discussed in the last section of the paper. 

\section{State of the art}
\label{sec:soa}



\section{Wilcoxon tournament}
\label{sec:wilcoxon}



\section{Conclusions}



\section{Acknowledgments}

This work has been supported in part by project ANYSELF
(TIN2011-28627-C04-02 and -01).
The authors would like to thank the FEDER of European Union for
financial support via project "Sistema de Informaci�n y Predicci�n de
bajo coste y aut�nomo para conocer el Estado de las Carreteras en
tiempo real mediante dispositivos distribuidos" (SIPEsCa) of the
"Programa Operativo FEDER de Andaluc�a 2007-2013". We also thank all
Agency of Public Works of Andalusia Regional Government staff and
researchers for their dedication and professionalism.

\begin{figure}
\begin{center}
\includegraphics[width=6cm](logos_SIPESCA_2.jpg)
\end{center}
\end{figure}

\bibliographystyle{splncs03}
\bibliography{geneura,wilcoxon}  % sigproc.bib is the name of the Bibliography in this case

\end{document}
