\documentclass{llncs}
\usepackage[latin1]{inputenc}
\usepackage{graphicx}        % standard LaTeX graphics tool
\usepackage{url}
\usepackage{listings}

\begin{document}

\title{Taming noisy fitness using Wilcoxon statistical test}
\subtitle{}

\author{J.J. Merelo\inst{1}  \and Antonio
  Mora\inst{1}  \and Anna I. Esparcia-Alc�zar\inst{2} \and V�ctor Rivas\inst{3}}

\institute{University of Granada\\
       Department of Computer Architecture and Technology, ETSIIT\\
       18071 - Granada\\
       \email{\{jmerelo,pedro,amorag\}@geneura.ugr.es}
\and
S2Grupo\\
\email{aiesparcia@s2grupo.com}
\and
Universidad de M�laga\\
Departamento de Lenguajes y Sistemas Inform�ticos\\
\email{ccottap@lcc.uma.es}
}

\maketitle
\begin{abstract}
Noisy evaluation functions show up in may different optmization
problems, from industrial optimization to strategy games. Dealing with
them is not straightforward because of the inherent uncertainty in the
true value of the fitness of an individual, if it actually
exists. Several methods based on implicit or explicit average or
changes in selection have been proposed in the past, but they involve
a substantial redesign of the algorithm and the software used to solve
the problem. In this paper we propose a new method based on using
statistical tests to impose a partial order on the population; this
partial order is used to assign a fitness value to every individual
which can be used straitghforwarly in any selection function. 
Tests over a combinatorial optimization problem show that, despite
increasing computation time, the number of evaluations needed to reach
a solution is much smaller than the one needed by other methods that
use implicit or explicit averages for the fitness function. 
\end{abstract}

\section{Introduction}

Noise in fitness has different origins. It can be inherent to the
individual that is evaluated; for instance, in 
\cite{DBLP:journals/jcst/MoraFGGF12} a game-playing bot that includes a
set of application rates is optimized. This results in different
actions in different runs, and obviously different success rates and
then fitness. Even comparisons with other individuals can be affected:
given exactly the same pair of individuals, the chance of one beating
the other can vary in a wide range. In other cases like the one
presented in the MADE environment, where whole worlds are evolved
\cite{2014arXiv1403.3084G} the same kind of noisy environment will
happen. 

The examples mentioned above are actually one or the four categories
where uncertainties in fitness are found. These four types include also,
according to \cite{Jin2005303} approximated fitness functions
(originated by, for instance, surrogate models); robust functions,
where the main focus is in finding values with high tolerance to
change in initial evaluation conditions, and finally dynamic fitness
functions, where the {\em inherent} value of the function changes with
time, Our main interest will be in the first type, since it is the one
that we have actually met in the past and has led to the development
of this paper. 

At any rate, in this paper we will not be dealing with actual
problems; we will try to simulate the effect of noise on combinatorial
optimization functions using the same shape, and hopefully, amplitude,
that we actually have found in problems so far. In fact, from the
point of view of dealing with fitness, these are the main features of
noise we will be interested in. Besides, we will deal mainly with
additive noise with 0 mean and variance equal to 1.

The rest of the paper is organized as follows: next we describe the
state of the art in the treatment of noise in fitness functions. The
method we propose in this paper,  called Wilcoxon Tournament, will be
shown in Section \ref{sec:node}; experiments are described and
results shown in Section \ref{sec:dist} and its implications
discussed in the last section of the paper. 

\section{State of the art}
\label{sec:soa}



\section{Wilcoxon tournament}
\label{sec:wilcoxon}



\section{Conclusions}



\section{Acknowledgments}

This work has been supported in part by project ANYSELF
(TIN2011-28627-C04-02 and -01).
The authors would like to thank the FEDER of European Union for
financial support via project "Sistema de Informaci�n y Predicci�n de
bajo coste y aut�nomo para conocer el Estado de las Carreteras en
tiempo real mediante dispositivos distribuidos" (SIPEsCa) of the
"Programa Operativo FEDER de Andaluc�a 2007-2013". We also thank all
Agency of Public Works of Andalusia Regional Government staff and
researchers for their dedication and professionalism.

\begin{figure}
\begin{center}
\includegraphics[width=6cm](logos_SIPESCA_2.jpg)
\end{center}
\end{figure}

\bibliographystyle{splncs03}
\bibliography{geneura,wilcoxon}  % sigproc.bib is the name of the Bibliography in this case

\end{document}
